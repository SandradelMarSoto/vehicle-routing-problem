\documentclass{article}
\usepackage{arxiv}
\usepackage[utf8]{inputenc} 
\usepackage{float}
\usepackage{listings}
\usepackage[T1]{fontenc}    
\usepackage{hyperref}       
\usepackage{url}            
\usepackage{mathtools}
\usepackage{amssymb,mathrsfs}
\usepackage{booktabs}       
\usepackage{amsfonts}       
\usepackage{dsfont}   
\usepackage[ruled,vlined]{algorithm2e}
\usepackage[spanish]{babel}

\title{Problema de Enrutamiento de Veh\'iculos}
\author{
  Sandra del Mar Soto Corderi\\
  No. cuenta: 315707267
}
\date{}

\begin{document}
\maketitle

\section{Introducción}

El problema del que se hablará en este reporte es el de enrutamiento de vehículos mejor conocido como \textbf{VRP}. 

El VRP implica enrutar una flota de vehículos, cada uno de ellos visitando un conjunto de nodos de modo que cada nodo sea visitado exactamente por un vehículo solo una vez. Entonces, el objetivo es minimizar la distancia total recorrida por todos los vehículos.



\begin{thebibliography}{9}
	\bibitem{hybrid}
	Yousefikhoshbakht, M. and Khorram, E., 2012. Solving the vehicle routing problem by a hybrid meta-heuristic algorithm. Journal of Industrial Engineering International, 8(1).
	
	\bibitem{ant}
	Mazzeo, S., \& Loiseau, I., 2004. An ant Colony algorithm for The capacitated vehicle routing. Electronic Notes in Discrete Mathematics 18. 181–186.
\end{thebibliography}

Si hay duda de alguna fuente favor de contactarme para proporcionarla.

\end{document}